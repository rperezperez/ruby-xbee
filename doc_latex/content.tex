% Special character reminder: # $ % & ~ _ ^ \ { }
\setlength{\parindent}{0pt}
\section{Hardware for testing}
\begin{table}[ht]
\caption{Xbees used for testing}
\centering
\begin{tabular}{c c c c}
\hline\hline
Model & Operating Model & Function Set & Firmware \\ [0.5ex] % inserts table %heading
\hline
XB24-Z7WIT-004 & API & Zigbee Coordinator API & 21A7 \\ [1ex]
\hline
\end{tabular}
\label{table:nonlin}
\end{table}

\section{Known Errors and Issues}
\subsection{No more config/xbeeconfig.rb}\smalltodo[noline]{Find replacement solution for xbeeconfig.rb}
The config/xbeeconfig.rb was removed by Mike Ashmore and for good reason. The config had needed configuration detailing where and how to read the XBee module, but it wasn't the best solution especially as it ruined a good way to Gemify the Ruby::XBee code.

\noindent The commit code reads: \textit{"Removed conf/xbeeconfig.rb - it's not a good way to do this. Of cours… …e now I need to figure out what \textbf{is} a good way to do this.} However, seems there was no good solution as currently the values are hardcoded to \textbf{bin/ruby-xbee.rb}.

\subsection{Better exception handling}
\smalltodo[noline]{Implement critical exception handling for proper functioning}
Currently ruby-xbee is very easily choked. Needs to be fixed.

\section{Testing}
\subsection{Wha..?}
Ruby in general is quite famous for it's Test::Unit but also for Rspec and Cucumber - in short, it's a thing to do when developing Ruby/Rails app and organizing an orchestra of good tests can keep headache away later on.

\section{What I'm in the progress of doing}
\begin{itemize}
\item Fixing for Bundle and Rake, integration with Travis and Awesome testing!
\end{itemize}

\section{Memorable quotes picked up along the Journey}
\epigraph{Practically all the software in the world is either broken or very difficult to use. So users dread software. They've been trained that whenever they try to install something, or even fill out a form online, it's not going to work. \textit{I} dread installing stuff and I have a Ph.D. in computer science.}{Paul Graham, \textit{Founders at Work} \cite{rubytutorial}}